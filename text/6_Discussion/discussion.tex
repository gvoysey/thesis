% -*- root: ../gvoysey-thesis.tex -*-
\chapter{Discussion}
\label{chapter:Discussion}
\thispagestyle{myheadings}

% set this to the location of the figures for this chapter. it may
% also want to be ../Figures/2_Body/ or something. make sure that
% it has a trailing directory separator (i.e., '/')!
\graphicspath{{6_Discussion/Figures/}}
\section{Chapter Summary} % (fold)
\label{sec:discussion_summary}
This chapter compares the results obtained in this thesis with the human results obtained by \citeauthor{Mehraei2016Auditory}, and offers justifications and possible explanations for their similarities and differences.
% section discussion_summary (end)
\iffalse
	\section{Comparison of Results to Prior Work} % (fold)
	\label{sec:comparison_of_results_to_prior_work}
	\subsection{Prior Models} % (fold)
	\label{sub:prior_models}

	% subsection prior_models (end)
	\subsection{Prior Experimental Results} % (fold)
	\label{sub:prior_experimental_results}
\fi 

\section{Nonlinear Behaviors in the Verhulst Model} % (fold)
\label{sec:nonlinear_behaviors_in_the_verhulst_model}
During the course of this work, an unexpected phenomenon was observed in the behavior of the Verhulst model in its response to stimuli of long duration.  In response to a sustained pure tone stimulus, the model predicts a strong response along the sections of the basilar membrane near the frequency of the pure tone, consistent with intuition. Further, the model predicts small amplitude BM displacement at higher frequencies, correctly reflecting dispersion of energies along the BM.  However, at the level of the IHC synapse, the off-frequency firing rate estimates are several times larger than the on-frequency response, and fall outside physiological boundaries. This behavior is not consistent with the BM displacement predictions of the previous stage of the model.  

To relate basilar membrane displacement to IHC firing rates, the Verhulst model implements a three-store synaptic diffusion model adapted from~\cite{Westerman1988Diffusion} and extends it to have place-dependent initial values of vesicle state.   Following~\cite{Liberman1978AuditoryNerve}, the saturated firing rate of a hair cell was also adapted to be place-dependent and used as a reset threshold for the diffusion model parameters.  It is possible that in certain situations, this threshold is never reached and thus the firing rate estimate grows disproportionately, leading to the observed large-magnitude response at high frequencies to a low frequency tone.

This behavior would potentially overestimate the off-frequency basal (high frequency) response to a sustained, more apical (low frequency) stimulus.  However, some evidence exists \citep{Kiang1974Tails,Yates1990Basilar} that basal responses to apical stimuli can approach threshold in some cases, so a prediction of supra-threshold firing rate at high frequencies to a low frequency tone may not be \emph{a priori} incorrect.
% section nonlinear_behaviors_in_the_verhulst_model (end)

\section{Consequences of AN Population Response Modeling} % (fold)
\label{sec:consequences_of_percentage_weighting_degradation_for_synaptopathy}
To obtain the total contribution of one inner hair cell, and thus one CF, to the population response of the AN, the model scales the responses of a low-, mid-, and high-spontaneous rate modeled fiber by three linear weights, thus reflecting what proportion of spiral ganglia belong to a given category for that IHC.  This approach makes two interrelated assumptions.  

First, it assumes that the spontaneous behavior of a given fiber is sufficiently similar to that of all others of its spontaneous rate category that it is not necessary to simulate each fiber individually.  In the case of the Verhulst model, this assumption is realistic since the model considers spontaneous rates to be fixed per fiber type.  However, the Zilany model may be configured so that estimates of spontaneous rate contain additive white Gaussian noise with a different random seed for every simulation, so the firing statistics of a given fiber may differ both from others of its spontaneous rate class and from itself over sustained periods or repeated simulations.

Second, as a result of the stochasticity of the Zilany model, it would potentially be informative to investigate the loss of individual fibers in a Monte Carlo simulation to address the variance in model responses.  This would further complicate simulation and increase the dimensionality of \emph{post-hoc} analysis. 

These assumptions make computation of AN responses practical: only three fibers per CF are modeled.  Using the default parameters that were used in this work, 3,000 fibers were simulated per model iteration.  Simulating each fiber individually with individual stochastic spontaneous rates would incur a tenfold increase in the number of fibers to simulate, suggesting that a full exploration of the parameter space, as was done in this work, would take approximately 90 days to compute.

At the same time, it would more accurately reflect the consequences of cochlear synaptopathy.  To the extent that the random noise in a fiber's spontaneous rate is orthogonal to that of any other fiber of the AN, and to the extent that this noise has random phase, any individual fiber will contribute a different amount to the compound action potential of the AN and its loss is not well represented by the current approach.  
% section consequences_of_percentage_weighting_degradation_for_synaptopathy (end)

\section{Nonlinear Synaptopathic Models} % (fold)
\label{sec:nonlinear_synaptopathic_models}
The unexpectedly small effects of modeled synaptopathy on the overall model output may in part be due to the uniformity of the synaptopathic impairment that was simulated.  While modeled impairment was specific to fibers of different spontaneous rates, it was applied uniformly over all CFs, as the variability of fiber type distribution per CF would impair some frequency ranges more than others for a given neuropathic condition. 

However, sensorineural hearing loss, particularly age-related hearing loss, is often specific to high frequencies while leaving low frequency bands largely unchanged.   Noise-induced or ototoxic hearing loss may have a narrower frequency band, leading to a notched audiogram while leaving other frequencies at normal thresholds, and models of synaptopathy that reflect these more complex losses may have more complex effects on simulation output.

Because the ABR arises from the synchronous activity of entire nerves or brainstem or midbrain areas, a frequency selective perturbation of the output of the AN should produce an effect of greater magnitude than the synaptopthies modeled in this work. 

The minimal changes in Wave I peak amplitude are expected.
\citeauthor{Liberman2014Efferent} and others have demonstrated the robustness of audiometric thresholds in animals with as little as 20\% of the original hair cell population intact, so the preservation of the AN compound action potential is consistent even with very severe synaptopathies.  Wave I peak amplitudes will also vary with the stimulus.  
% section nonlinear_synaptopathic_models (end)