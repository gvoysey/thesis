% -*- root: ../gvoysey-thesis.tex -*-
\chapter{Discussion}
\label{chapter:Discussion}
\thispagestyle{myheadings}

% set this to the location of the figures for this chapter. it may
% also want to be ../Figures/2_Body/ or something. make sure that
% it has a trailing directory separator (i.e., '/')!
\graphicspath{{6_Discussion/Figures/}}
\section{Chapter Summary} % (fold)
\label{sec:discussion_summary}
This chapter compares the results obtained in this thesis with the human results obtained by \citeauthor{Mehraei2016Auditory}, and offers justifications and possible explanations for their similarities and differences.
% section discussion_summary (end)

\section{Comparison of Results to Prior Work} % (fold)
\label{sec:comparison_of_results_to_prior_work}
\subsection{Prior Models} % (fold)
\label{sub:prior_models}

% subsection prior_models (end)
\subsection{Prior Experimental Results} % (fold)
\label{sub:prior_experimental_results}

% subsection prior_experimental_results (end)
% section comparison_of_results_to_prior_work (end)

\section{Nonlinear behaviors in the Verhulst Model} % (fold)
\label{sec:nonlinear_behaviors_in_the_verhulst_model}
During the development of this work,sustained tones 
% section nonlinear_behaviors_in_the_verhulst_model (end)

\section{Consequences of percentage weighting degradation for synaptopathy} % (fold)
\label{sec:consequences_of_percentage_weighting_degradation_for_synaptopathy}
The methods used in this work to translate the IFR of one ANF of a given SR into the summed population response of the AN, as well as to model synaptopathy do not fully represent the stochastic nature of AN behavior. 
% section consequences_of_percentage_weighting_degradation_for_synaptopathy (end)

\section{Nonlinear Synaptopathic Models} % (fold)
\label{sec:nonlinear_synaptopathic_models}

% section nonlinear_synaptopathic_models (end)