% -*- root: ../gvoysey-thesis.tex -*-
\chapter{Conclusion}
\label{chapter:Conclusion}
\thispagestyle{myheadings}

% set this to the location of the figures for this chapter. it may
% also want to be ../Figures/2_Body/ or something. make sure that
% it has a trailing directory separator (i.e., '/')!
\graphicspath{{7_Conclusion/Figures/}}

\section{Conclusion} % (fold)
\label{sec:conclusion}
A model environment that allows the direct differential comparison of two different models of the auditory periphery, the auditory nerve, and two different models of the midbrain and brainstem was created and tested.  It incorporates new functionality in the form of a more sophisticated approximation of the population response of the auditory nerve that is aligned with recent anatomical and physiological work.  

Further, a tool to robustly explore the summed parameter space of all of the components of the modeling environment was implemented to allow modeling experiments to be reliably designed and run, and produce results that can be analyzed in any language and distributed with confidence. 

The utility of the modeling environment was shown in the exploration of the contributions of model parameter effects in the simulation of ABRs.  In comparison to human measurements in the same task, the addition of CF-weighted auditory nerve responses more closely matched the experimental results than prior models.
% section conclusion (end)

\section{Future Directions} % (fold)
\label{sec:future_directions}
\subsection{Modeling of Specific Hearing Loss Types} % (fold)
\label{sub:modeling_of_specific_hearing_loss_types}
An important step forward would be the incorporation of more complex synaptopathic degradations.  For example, those that result in audiometric threshold shifts, or have frequency-dependent losses as well as SR-dependent losses could provide new insights into the roles of off-frequency listening in a degraded auditory periphery.

\subsection{Anatomically Inspired Brainstem Contribution Adaptation} % (fold)
\label{sub:anatomically_inspired_brainstem_adaptation}
The proportion of band-pass, band-reject, and low-pass units in \citeauthor{Carney2015Speech}s model were held fixed throughout this work.  However, there is some anatomical evidence to suggest that selectivity in projections from the CN to IC could isolate ascending CN inputs to discrete areas of IC, where the distribution of MTF neuron types may be nonuniform.
