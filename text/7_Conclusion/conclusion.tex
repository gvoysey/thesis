% -*- root: ../gvoysey-thesis.tex -*-
\chapter{Conclusion}
\label{chapter:Conclusion}
\thispagestyle{myheadings}

% set this to the location of the figures for this chapter. it may
% also want to be ../Figures/2_Body/ or something. make sure that
% it has a trailing directory separator (i.e., '/')!
\graphicspath{{7_Conclusion/Figures/}}

\section{Summary} % (fold)
\label{sec:conclusion}
A model environment that allows the direct differential comparison of two different models of the auditory periphery, the auditory nerve, and two different models of the midbrain and brainstem was created and tested.  It incorporates new functionality in the form of a more sophisticated approximation of the population response of the auditory nerve that is aligned with recent anatomical and physiological work.  

Further, a tool to robustly explore the summed parameter space of all of the components of the modeling environment was implemented to allow modeling experiments to be reliably designed and run, and produce results that can be analyzed in any language and distributed with confidence. 

The utility of the modeling environment was shown in the exploration of the contributions of model parameter effects in the simulation of ABRs.  In comparison to human measurements in the same task, the addition of CF-weighted auditory nerve responses more closely matched the experimental results than prior models.
% section conclusion (end)

\section{Future Directions} % (fold)
The modeling environment created for this thesis allows a new level of flexibility to study the effects of complex pathologies and investigate theories of mechanisms of hearing impairment in the subcortical processing areas of the human nervous system.   Further, it provides a straightforward means by which component models may be easily updated and improved, and by which new models may be added.   

The following areas are those which are mostly likely to be immediately fruitful in continuing this work.
\label{sec:future_directions}
\subsection{Modeling of Specific Hearing Loss Types} % (fold)
\label{sub:modeling_of_specific_hearing_loss_types}
An important step forward would be the incorporation of more complex synaptopathic degradations.  For example, very severe synaptopathies such as those that result in audiometric threshold shifts, or cases where losses are specific not only to fiber type but also to CF could provide new insights into the roles of off-frequency listening in a degraded auditory periphery.

Another approach for investigating the role of low-SR fiber loss in HHL could involve the combination of CF weighting of fiber distribution with models of high-frequency hearing loss.  At the stimulus levels simulated in this work, frequency tuning along the BM is significantly less sharp than it is at threshold.  This broadening of tuning curves has significant implications if fiber loss sufficient to induce HHL occurs preferentially at high frequencies where low-SR fibers are prevalent.   In the context of speech in noise, these areas of the cochlea may be thought to be too high frequency to significantly affect speech coding, but at supra-threshold levels, this may not the case. Consequently, if those same areas synapse with a higher percentage of low-SR, slower-saturating fibers which are preferentially damaged by noise and age, the effects on signal detection in real world environments may be significant. 

If high frequency losses are shown to produce degraded representations of speech, this would offer interesting evidence in the possible role of low-SR fibers in nominally off-frequency listening.  At sound levels routinely encountered in daily life, frequency tuning curves are quite broad.  Potentially, this would allow high frequency areas, which have more low-SR fibers that saturate more slowly, to contribute to AN responses to speech with most energy at lower frequencies.  High frequency hearing loss that preferentially damaged low-SR fibers would disrupt that hypothetical mechanism.  The modeling environment created for this work could drive prediction and experimental design to probe this question. 

\subsection{Anatomically Inspired IC Weighting} % (fold)
\label{sub:anatomically_inspired_brainstem_adaptation}
The proportion of band-pass, band-reject, and low-pass units in~\cite{Carney2015Speech} were held fixed throughout this work.  However, there is some anatomical evidence to suggest that selectivity in projections from the CN to IC could isolate ascending CN inputs to discrete areas of IC, where the distribution of neural response properties may be nonuniform.  This anatomically driven specificity could impose nonuniform latencies on frequency-specific portions of the AN compound action potential, which has the potential to substantially affect Wave V delays.

With the increasingly detailed physiological tools now available, were such inhomogeneities to be characterized, they could be modeled by adapting the IC input weights to more accurately reflect the behavior of the IC to ascending input. 
