% -*- root: ../gvoysey-thesis.tex -*-
\chapter{Aims}
\label{chapter:Aims}
\thispagestyle{myheadings}
This thesis investigates several models of the peripheral and central auditory systems and the utility of their predictive abilities for cochlear synaptopathy in simulations of human audition. 

Three aims were established.  First, to develop a coherent modeling environment that combines models of the middle ear, auditory nerve, and auditory brainstem and midbrain from~\cite{Zilany2014Updated,Verhulst2015Functional,Nelson2004Phenomenological,Carney2015Speech} into one software package where the utility of each model could be compared head to head.  Second, to advance the state of the models of the auditory periphery by extending them with new capabilities supported by available anatomical and physiological research.  Third, to use the developed tool to explore the proposed mechanisms underlying psychophysical and large-scale electrophysiological studies of cochlear synaptopathy with higher fidelity.


\section{Aim I. Simulate the ABR Response of a Noise-Masking Task with Variable SR Contributions and Model Parameters}
A modeling environment was created.  It incorporates two peripheral models of the auditory system: the Zilany model with humanized parameters \citep{Zilany2014Updated} and the Verhulst model \citep{Verhulst2015Functional}.  

The Cochlea modeling environment \citep{Rudnicki2014Cochlea} was used to provide easy incorporation of the Zilany model into the new modeling environment. The transmission-line model of~\cite{Verhulst2015Functional}, which has the potential to perform better in broadband noise due to its  accounting for cochlear dispersion, was directly integrated.
At the conclusion of this aim, direct comparisons between the estimates of ABR Wave I and Wave V by~\cite{Zilany2014Updated} and~\cite{Verhulst2015Functional} were performed for a variety of experimental conditions.

\section{Aim II. Integrate Improved Brainstem Models}  

We hypothesized that the current approach to IC modeling taken in~\cite{Verhulst2015Functional,Mehraei2016Auditory} does not fully account for the responses to a low-SR knockout AN model, and consequently under-represents the effects on the ABR Wave V that have been experimentally measured.  

In particular, an extension of the approach currently taken by~\cite{Verhulst2015Functional} was presented by~\cite{Carney2015Speech}.  It provides multiple classes of IC neurons that were shown to track complex tones (vowel formants) in noise.  To guide the selection of model weights and connectivities, relevant neuroanatomical literature were consulted.  Crucially, studies by~\cite{Ryugo2008Projections} and others have shown selectivities in SR projections to the small cap of the DCN, which will guide  our modeling work by introducing specificities in weighting. 

Further, while the latency change trend is preserved between both the models proposed by \citeauthor{Zilany2014Updated} and \citeauthor{Verhulst2015Functional}, the magnitude of the effect is greatly different.  This discrepancy may be remedied by introduction of new IC modeling components, which are better incorporated in the~\cite{Zilany2014Updated} model

\section{Aim III. Relate Model Responses to Psychophysical Measures}  
We will compare subject Wave V latency data from \citeauthor{Mehraei2015Auditory} and \citeauthor{Mehraei2015Individual} as ground truth to the improved model output.  Interpreting the relative effects of different modeling parameters may elucidate which aspects of the auditory periphery are important in the further study of cochlear synaptopathy and its contribution to hidden hearing loss. 