% BU ECE template for MS thesis and PhD dissertation.
%
%==========================================================================%
% MAIN PREAMBLE 
%==========================================================================%
\documentclass[12pt]{report}          % Single-sided printing for the library
%\documentclass[12pt,twoside]{report} % Double-sided printing
%nicely formatted equations.
\usepackage[intlimits]{amsmath}
\usepackage{amsfonts,amssymb}
\DeclareSymbolFontAlphabet{\mathbb}{AMSb}


%% URLs shouldn't have stupid fonts.
\usepackage{url}
\urlstyle{same}
%% standard options inserted by the template
\usepackage{float,subfigure}
\usepackage[bf]{caption}       
\setcaptionmargin{0.5in}
\usepackage{fancyheadings,fancybox,ifthen}
\usepackage{bu_ece_thesis}
\usepackage{lscape,afterpage}
\usepackage{xspace}
\usepackage{color}
\usepackage{pdfpages}
\usepackage{parskip}
\usepackage{siunitx}
\usepackage{cleveref}
\sisetup{output-exponent-marker=\ensuremath{\mathrm{e}}}
%==========================================================================%
%%% graphicx and pdf creation
\usepackage{graphicx}
\definecolor{gray}{RGB}{64,64,64}
\definecolor{mediumBlue}{RGB}{0,0,205}
%\usepackage{psfrag}
\DeclareGraphicsExtensions{.eps}   % extension for included graphics
%\usepackage{thumbpdf}              % thumbnails for ps2pdf
\usepackage[                       % hyper-references for ps2pdf
colorlinks=true,
%colorlinks=false,
linkcolor=gray,
citecolor=gray,
bookmarks=true,%                   % generate bookmarks ...
bookmarksnumbered=true,%           % ... with numbers
%hypertexnames=false,%              % needed for correct links to figures !!!
breaklinks=true,%                  % breaks lines, but links are very small
%linkbordercolor={0,0,0},%          % blue frames around links
%pdfborder={0 0 112.0}				% border-width of frames 
]{hyperref}%  
%                                   % will be multiplied with 0.009 by ps2pdf
 \hypersetup{
  pdfauthor   = {Graham Voysey <gvoysey@bu.edu>},
  pdftitle    = {Development of a Flexible Modeling Environment for Evaluating Subcortical Auditory Systems.pdf},
  pdfsubject  = {Master's thesis},
  pdfkeywords = {biomedical engineering, hearing},
  pdfcreator  = {LaTeX/hyperref},
  pdfproducer = {LaTeX/P}
 }
%bibliography configuration
\include{bib_preamble}
%==========================================================================%
% customized commands can be placed here
%\newcommand{\figref}[1]{Figure~\ref{#1}}
%\newcommand{\chapref}[1]{Chapter~\ref{#1}}
%\newcommand{\latex}{\LaTeX\xspace}
%==========================================================================%
\renewcommand*{\chapterautorefname}{Chapter}
\renewcommand*{\sectionautorefname}{Section}
%==========================================================================%
% BEGIN
%==========================================================================%
\begin{document}

% The preliminary pages
% This file contains all the necessary setup and commands to create
% the preliminary pages according to the buthesis.sty option.

\title{TBD}

\author{Graham Voysey}

% Type of document prepared for this degree:
%   1 = Master of Science thesis,
%   2 = Doctor of Philisophy dissertation.
%   3 = Master of Science thesis and Doctor of Philisophy dissertation.
\degree=1

\prevdegrees{B.S., Boston University, 2006}

\department{Department of Biomedical Engineering}

% Degree year is the year the diploma is expected, and defense year is
% the year the dissertation is written up and defended. Often, these
% will be the same, except for January graduation, when your defense
% will be in the fall of year X, and your graduation will be in
% January of year X+1
\defenseyear{2015}
\degreeyear{2015}

% For each reader, specify appropriate label {First, second, third},
% then name, then title. Warning: If you have more than five readers
% you are out of luck, because it will overflow to a new page.
% Sometimes you may wish to put part of the title in with the name
\reader{First}{H. Steven Colburn, PhD}{Professor of Biomedical Engineering}
\reader{Second}{Barbara Shinn-Cunningham, PhD}{Professor of Biomedical Engineering}
\reader{Third}{Allyn E. Hubbard, PhD}{Professor of Electrical Engineering}

% The Major Professor is the same as the first reader, but must be
% specified again for the abstract page
\majorprof{First M. Last, PhD}{\mbox{Department of Biomedical Engineering}, \mbox{secondary appointment}}

%                       PRELIMINARY PAGES
% According to the BU guide the preliminary pages consist of:
% title, copyright (optional), approval,  acknowledgments (opt.),
% abstract, preface (opt.), Table of contents, List of tables (if
% any), List of illustrations (if any). The \tableofcontents,
% \listoffigures, and \listoftables commands can be used in the
% appropriate places. For other things like preface, do it manually
% with something like \newpage\section*{Preface}.

% This is an additional page (do not hand it in at the library) to print
% boxed-in title, author and degree statement so that they are visible through
% the opening in BU covers used for reports. This makes a nicely bound copy.
%\buecethesistitleboxpage

% Make the titlepage based on the above information.  If you need
% something special and can't use the standard form, you can specify
% the exact text of the titlepage yourself.  Put it in a titlepage
% environment and leave blank lines where you want vertical space.
% The spaces will be adjusted to fill the entire page.
\maketitle
\cleardoublepage

% The copyright page is blank except for the notice at the bottom. You
% must provide your name in capitals.
\copyrightpage
\cleardoublepage

% Now include the approval page based on the readers information
\approvalpage
\cleardoublepage

% Here goes your favorite quote.
\newpage
\thispagestyle{empty}
% -*- root: ../gvoysey-thesis.tex -*-
\phantom{.}
\vspace{4in}

\begin{singlespace}
\begin{quote}
  We've heard a lot of models\ldots{}and heard suggested that we should take things out of our models to figure out what's important.  But in some sense, when I look at the diversity of models that have been presented so far---each of us leave \emph{out} things.  So maybe in some sense we've got a start towards that approach.\\  
  
  So I can ask this question two ways, but let me ask it this way: \emph{What should we leave in?} What's the bare minimum we should leave in as we try to understand what's important about the function of the cochlea?
    % \textit{Facilis descensus Averni;}\\
  % \textit{Noctes atque dies patet atri janua Ditis;}\\*
  % \textit{Sed revocare gradum, superasque evadere ad auras,}\\
  % \textit{Hoc opus, hic labor est.}\hfill{Virgil (from Don's thesis!)}
  \vspace{2.5em}
  \begin{flushright}
  \textit{David C. Mountain\\Mechanics of Hearing (Attica, Greece 2014)}
  \end{flushright}
\end{quote}
\end{singlespace}
  

% \vspace{0.7in}
%
% \noindent
% [The descent to Avernus is easy; the gate of Pluto stands open night
% and day; but to retrace one's steps and return to the upper air, that
% is the toil, that the difficulty.]

\cleardoublepage

% The acknowledgment page should go here. Use something like
% \newpage\section*{Acknowledgments} followed by your text.
\newpage
\section*{\centerline{Acknowledgments}}
% -*- root: ../gvoysey-thesis.tex -*-
To David Mountain, who saw something worthy in me. 

\cleardoublepage

% The abstractpage environment sets up everything on the page except
% the text itself.  The title and other header material are put at the
% top of the page, and the supervisors are listed at the bottom.  A
% new page is begun both before and after.  Of course, an abstract may
% be more than one page itself.  If you need more control over the
% format of the page, you can use the abstract environment, which puts
% the word "Abstract" at the beginning and single spaces its text.

\begin{abstractpage}
% -*- root: ../gvoysey-thesis.tex -*-
% ABSTRACT

\end{abstractpage}
\cleardoublepage

% Now you can include a preface. Again, use something like
% \newpage\section*{Preface} followed by your text

% Table of contents comes after preface
\tableofcontents
\cleardoublepage

% If you have tables, uncomment the following line
%\listoftables
%\cleardoublepage

% If you have figures, uncomment the following line
\newpage
\listoffigures
\cleardoublepage

% List of Abbrevs is NOT optional (Martha Wellman likes all abbrevs listed)
\chapter*{List of Abbreviations}
\begin{center}
  \begin{tabular}{lll}
    \hspace*{2em} & \hspace*{1in} & \hspace*{4.5in} \\
    CAD  & \dotfill & Computer-Aided Design \\
    CO   & \dotfill & Cytochrome Oxidase \\
    DOG  & \dotfill & Difference Of Gaussian (distributions) \\
    FWHM & \dotfill & Full-Width at Half Maximum \\
    LGN  & \dotfill & Lateral Geniculate Nucleus \\
    ODC  & \dotfill & Ocular Dominance Column \\
    PDF  & \dotfill & Probability Distribution Function \\
    $\mathbb{R}^{2}$  & \dotfill & the Real plane \\
  \end{tabular}
\end{center}
\cleardoublepage

% END OF THE PRELIMINARY PAGES

\newpage
\endofprelim
        
\cleardoublepage{}

\chapter{Introduction}
\label{chapter:Introduction}
\thispagestyle{myheadings}

\section{A brief history}
\label{sec:history}

Let's get started.

\cleardoublepage{}

% -*- root: ../gvoysey-thesis.tex -*-
\chapter{Literature Review}
\label{chapter:literaturereview}
\thispagestyle{myheadings}
% set this to the location of the figures for this chapter. it may
% also want to be ../Figures/2_Body/ or something. make sure that
% it has a trailing directory separator (i.e., '/')!
\graphicspath{{2_LiteratureReview/Figures/}}
\section{Chapter Summary} % (fold)
\label{sec:review_summary}
This chapter lays out a review of the relevant literature this thesis relies on. First, an overview of the clinical significance and relevant neuroanatomy of cochlear synaptopathy are given.  Then, a review of the computational models that will be used is presented. 

% section review_summary (end)
\section{Cochlear Synaptopathy} % (fold)
\label{sec:cochlear_synaptopathy}
% section cochlear_synaptopathy (end)
Deafferentiation is the loss of one or more synapses between an Inner Hair Cell (IHC) and its innervating spiral ganglia. 

\cite{Kujawa2009Adding} showed in noise exposed mice that significant deafferentiation can occur with no permanent changes in threshold tuning curves and no hair cell death. \autoref{fig:furman-2} shows that deafferentiation was confirmed histologically by triple-staining cross-sections of the Organ of Corti post noise exposure.  This reveals a synaptic loss. 

\begin{figure}[htbp]
	\centering
	\includegraphics[width=0.95\textwidth]{furman-2.pdf}
	\caption[Deafferentation of Inner Hair Cells]{Deafferentiation of IHCs precedes hair cell death in noise exposed mice.  Figure is reprinted from~\cite{Furman2013NoiseInduced}}
	\label{fig:furman-2}
\end{figure}


\cite{Sergeyenko2013AgeRelated} extended this work to demonstrate that this deafferentiation may also arise solely as a function of time.  In a study of mice aged 4 to 144 weeks that were never exposed to loud sounds, a similar loss of hair cell projections was observed.


\section{Physiology of the Auditory Nerve} % (fold)
\label{sec:physiology_of_the_auditory_nerve}
\subsection{Spontaneous Rates of Fibers} % (fold)
\label{sub:spontaneous_rates_of_fibers}
In the absence of stimulus, individual fibers of the auditory nerve exhibit a wide range of average firing rates: human AN fibers have spontaneous rates between 0--120 spikes/second.  Any individual fiber's spontaneous rate varies slowly over time, but will fall within a relatively narrow band.  The fibers of the auditory nerve are divided into two, or sometimes three, categories: low-, medium-, and high spontaneous rate (SR).  Different authors assign different maximum firing rates to each category:~\cite{Temchin2008Threshold} categorizes SRs below 18 spikes/second to be ``low/medium'', and anything above that to be ``high''.  Others, such as~\cite{Liberman1978AuditoryNerve}, define only two categories.  
% subsection spontaneous_rates_of_fibers (end)
\subsection{Low Spontaneous Rate Fibers Suffer Selective Losses} % (fold)
\label{sub:low_spontaneous_rate_fibers_suffer_selective_losses}
\cite{Furman2013NoiseInduced} and others have demonstrated that noise-induced cochlear synaptopathy is selective for low-SR fibers, particularly at high frequencies.  As shown in \autoref{fig:furman-5}, examination of fiber loss after acoustic trauma demonstrates a preferential loss of low-SR fibers, particularly above 4 kHz.  

\begin{figure}[htbp]
	\centering
	\includegraphics[width=0.95\textwidth]{furman-5.pdf}
	\caption[Synaptopathy is Selective]{Synaptopathy is selective for fibers with low spontaneous rates, and particularly selective for low-SR fibers at high frequencies. Figure reprinted from~\cite{Furman2013NoiseInduced}}
	\label{fig:furman-5}
\end{figure}

\section{Relevant Functional Neuroanatomy of the Auditory Midbrain} % (fold)
\label{sec:relevant_functional_neuroanatomy_of_the_auditory_midbrain}
\subsection{The Cochlear Nucleus} % (fold)
\label{sub:the_cochlear_nucleus}
The primary projection from the AN is the Cochlear Nucleus, an inhomogenous structure that is the first auditory relay station located in the ipsilateral medulla of the brainstem.  
\subsection{The Dorsal Cochlear Nucleus} % (fold)
\label{sub:the_dorsal_cochlear_nucleus}
\cite{Ryugo2008Projections} demonstrated in cat that low-SR fibers have a anatomical projection bias towards the small cap of the Dorsal Cochlear Nucleus.  While low-SR fibers project to many areas, the small cap receives input from low-SR fibers exclusively, suggesting a selective role for low-SR projections. \cite{Liberman1993Central} found similar results. 

Further, while projections are selective as shown in \autoref{fig:ryugo}, the projections have relatively shallow but broad arbors. This anatomical specificity of projection combined with a breadth of coverage supports a particular role for low-SR fibers, and does not rule out the possibility of further specificities in higher brainstem and midbrain areas. 

\begin{figure}[htbp]
	\centering
	\includegraphics[width=0.95\textwidth]{ryugo-1.pdf}
	\caption[Low SR Fibers Project to the Small Cap]{Low-SR fibers project to the small cap of the DCN.  From~\cite{Ryugo2008Projections}, this shows a lateral view of a low SR fiber as it collateralizes (red) in the rostral and lateral SCC (CF=0.45 kHz; SR=1.2 s/s; Th=34 dB SPL).}
	\label{fig:ryugo}
\end{figure}

While the VCN is critically important for the processing of binaural phenomena, the specificities of projection observed in the DCN suggest an interesting role for synaptopathic losses that may be more monaural.
% subsection the_ventral_cochlear_nucleus (end)
\subsection{The Inferior Colliculus} % (fold)
\label{sub:the_inferior_colliculus}
The IC has long been regarded as the last pre-thalamic obligate waystation for ascending auditory information, and a major center of pre-cortical auditory processing with many diverse functions \citep{Cant2005Atlas,Covey2008Inputs,Moore1985Projections}.~\cite{Beebe2016Extracellular} has recently found at least four morphologically different GABAergic neuron types in IC that react in different ways to auditory stimuli.
% subsection the_inferior_colliculus (end)
% section functional_neuroanatomy_of_the_auditory_midbrain (end)

\section{Models of the Auditory Periphery} % (fold)
\label{sec:models_of_the_auditory_periphery}
\subsection{The Verhulst Model} % (fold)
\label{sub:the_verhulst_model}
A functional model of the auditory periphery was developed by~\cite{Verhulst2015Functional}.  As outlined in \autoref{fig:verhulst-1}, the model consists of a middle ear preprocessing model adapted from~\cite{Meddis2010Computational}.  Input is passed to a cochlear transmission line model, which estimates BM displacements and velocities for an arbitrary number of BM sections (default: 1000).  Motions of the BM are translated into IHC bundle deflections and passed through a nonlinearity.  Estimates of the Instantaneous Firing Rate (IFR) are made by a method adapted from~\cite{Westerman1988Diffusion}, which implements a three-store diffusion model of synaptic vesicle and neurotransmitter release and reuptake.  

The version of the model given by~\cite{Verhulst2015Functional} also includes a CN and IC modeling stage from \cite{Nelson2004Phenomenological}, and the final model output are estimates of ABR Wave I, Wave III, and Wave V. 

\begin{figure}[htbp]
	\centering
	\includegraphics[width=0.95\textwidth]{verhulst-1.pdf}
	\caption[The Verhulst Model]{Major subunits of the Verhulst transmission line model of the auditory periphery.  Figure is reprinted from \cite{Verhulst2015Functional}.}
	\label{fig:verhulst-1}
\end{figure}


\subsection{The Zilany and Bruce Model} % (fold)
\label{sub:the_zilany_and_bruce_model}
\cite{Zilany2006Modeling} proposed a phenomenological, signals-driven model of the auditory periphery.  It has been refined and updated since to account for an increasing number of phenomena including estimations of speech intelligibility \citep{Zilany2007Predictions}, long-term IHC adaptation with power-law dynamics \citep{Zilany2009Phenomenological}, and updates to more closely model human parameters \citep{Zilany2014Updated}.  

The approach is outlined in \autoref{fig:zilany-1} and consists of a phenomenological power-law model that has filters for each stage of the periphery.  Fractional Gaussian noise is optionally added per-channel to simulate the stochasticities inherent in AN fiber spontaneous rates. 

\begin{figure}[htbp]
	\centering
	\includegraphics[width=0.95\textwidth]{zilany-1.pdf}
	\caption[The Zilany Model]{Major components of the Zilany model of the auditory periphery.  Figure reprinted from \cite{Zilany2009Phenomenological}}
	\label{fig:zilany-1}
\end{figure}
% subsection the_zilany_and_bruce_model (end)
\section{Models of the Auditory Midbrain and Brainstem} % (fold)
\label{sec:models_of_the_auditory_midbrain_and_brainstem}
\subsection{The Nelson-Carney Model} % (fold)
\label{sub:the_nelson_carney_model}
\cite{Nelson2004Phenomenological}

\subsection{The Carney Model} % (fold)
\label{sub:the_carney_model}

% subsection the_carney_model (end)
% subsection the_nelson_carney_model (end)
% section models_of_the_auditory_midbrain_and_brainstem (end)



\section{Candidate Objective Measures of Cochlear Synaptopathy} % (fold)
\label{sec:objective_measures_of_cochlear_synaptopathy}

% section objective_measures_of_cochlear_synaptopathy (end)

\cleardoublepage{}

% -*- root: ../gvoysey-thesis.tex -*-
\chapter{Aims}
\label{chapter:Aims}
\thispagestyle{myheadings}
This thesis investigates several models of the peripheral and central auditory systems and the utility of their predictive abilities for cochlear synaptopathy in simulations of human audition. 

Three aims were established.  First, to develop a coherent modeling environment that combines models of the middle ear, auditory nerve, and auditory brainstem and midbrain from~\cite{Zilany2014Updated,Verhulst2015Functional,Nelson2004Phenomenological,Carney2015Speech} into one software package where the utility of each model could be compared head to head.  Second, to advance the state of the models of the auditory periphery by extending them with new capabilities supported by available anatomical and physiological research.  Third, to use the developed tool to explore the proposed mechanisms underlying psychophysical and large-scale electrophysiological studies of cochlear synaptopathy with higher fidelity.


\section{Aim I. Simulate the ABR Response of a Noise-Masking Task with Variable SR Contributions and Model Parameters}
A modeling environment was created.  It incorporates two peripheral models of the auditory system: the Zilany model with humanized parameters \citep{Zilany2014Updated} and the Verhulst model \citep{Verhulst2015Functional}.  

The Cochlea modeling environment \citep{Rudnicki2014Cochlea} was used to provide easy incorporation of the Zilany model into the new modeling environment. The transmission-line model of~\cite{Verhulst2015Functional}, which has the potential to perform better in broadband noise due to its  accounting for cochlear dispersion, was directly integrated.
At the conclusion of this aim, direct comparisons between the estimates of ABR Wave I and Wave V by~\cite{Zilany2014Updated} and~\cite{Verhulst2015Functional} were performed for a variety of experimental conditions.

\section{Aim II. Integrate Improved Brainstem Models}  

We hypothesized that the current approach to IC modeling taken in~\cite{Verhulst2015Functional,Mehraei2016Auditory} does not fully account for the responses to a low-SR knockout AN model, and consequently under-represents the effects on the ABR Wave V that have been experimentally measured.  

In particular, an extension of the approach currently taken by~\cite{Verhulst2015Functional} was presented by~\cite{Carney2015Speech}.  It provides multiple classes of IC neurons that were shown to track complex tones (vowel formants) in noise.  To guide the selection of model weights and connectivities, relevant neuroanatomical literature were consulted.  Crucially, studies by~\cite{Ryugo2008Projections} and others have shown selectivities in SR projections to the small cap of the DCN, which will guide  our modeling work by introducing specificities in weighting. 

Further, while the latency change trend is preserved between both the models proposed by~\cite{Zilany2014Updated} and~\cite{Verhulst2015Functional}, the magnitude of the effect is greatly different.  This discrepancy may be remedied by introduction of new IC modeling components, which are better incorporated in the~\cite{Zilany2014Updated} model

\section{Aim III. Relate Model Responses to Psychophysical Measures}

We will compare subject Wave V latency data from \cite{Mehraei2015Auditory} and \cite{Mehraei2015Individual} as ground truth to the improved model output.  Interpreting the relative effects of different modeling parameters may elucidate which aspects of the auditory periphery are important in the further study of cochlear synaptopathy and its contribution to hidden hearing loss. 
\cleardoublepage{} 

% -*- root: ../gvoysey-thesis.tex -*-
\chapter{Methods}
\label{chapter:Methods}
\thispagestyle{myheadings}

% set this to the location of the figures for this chapter. it may
% also want to be ../Figures/2_Body/ or something. make sure that
% it has a trailing directory separator (i.e., '/')!
\graphicspath{{4_Methods/Figures/}}
\cleardoublepage{}

% -*- root: ../gvoysey-thesis.tex -*-
\chapter{Results}
\label{chapter:Results}
\thispagestyle{myheadings}

% set this to the location of the figures for this chapter. it may
% also want to be ../Figures/2_Body/ or something. make sure that
% it has a trailing directory separator (i.e., '/')!
\graphicspath{{5_Results/Figures/}}
\section{Chapter Summary} % (fold)
\label{sec:results_summary}
This chapter describes the results obtained when using the modeling environment described in~\autoref{chapter:Methods} to simulate a series of tone-in-noise experiments performed in humans by \citeauthor{Mehraei2016Auditory} to elucidate certain aspects of cochlear synaptopathy. 

% section results_summary (end)
\section{Tone in noise experiment} % (fold)
\label{sec:tone_in_noise}
The following parameters were 
\subsection{Stimuli} % (fold)
\label{sub:stimuli}
Following \citeauthor{Mehraei2015Auditory,Mehraei2016Auditory}, 6 stimului were programmatically generated and stored as WAV files with a sampling frequency of 100 kHz.  As shown in \autoref{fig:stimuli-used}, stimulus onset was delayed by 20 $\mu$s of silence, and then consisted of 80 dB SPL clicks with a repetition rate of 100 ms in the presence of gaussian noise at different signal to noise ratios. 

\begin{figure}[htbp]
	\centering
	\includegraphics[width=0.95\textwidth]{stimuli-used.pdf}
	\caption[Experimental Stimuli]{Stimuli used to drive the auditory models.}
	\label{fig:stimuli-used}
\end{figure}
% subsection stimuli (end)

\begin{itemize}
	\item Ran the same stimului used by~\cite{Mehraei2015Auditory}: 80dB click, varying SNR
	\item Ran 1208 model configurations which varied  the stimulus SNR, the level and type of synaptopathy applied to the results, which peripheral model was used, whether the BM had any cf-weighting, and which brainstem model was used. 
	\item used a python library and cluster resources to make sure the results are easily searchable. 
	\item roughly 400 GB of model results.
	\item Can now examine the effects of individual parameter changes. 
	\item can now compare results to both prior modeling results and human data. 
\end{itemize}


\section{Simulations}
The experiment design tool described in \autoref{sec:automated_parameter_exploration} was used to specify a range of values for each parameter in Corti to reveal the relative contributions of each. 

In total, 240 separate simulations were run in parallel on Boston University's high-performance computing cluster over the course of approximately 9 days.  Model output was automatically stored into a HDF5 database approximately 250 GB in size.

\section{Effect of peripheral model} % (fold)
\label{sec:effect_of_peripheral_model}
The verhulst and zilany models will produce different estimates of the auditory nerve response. Prior work had their results very different; with recent changes to the verhulst model, this may have changed.
% section effect_of_peripheral_model (end)

\section{Effect of Synaptopathy} % (fold)
\label{sec:effect_of_synaptopathy}
Six kinds of synaptopathy were simulated: uniform and low-- and medium--SR specific losses of 10, 25, and 50 percent of each fiber type. 
% section effect_of_synaptopathy (end)

\section{Effect of CF weighting} % (fold)
\label{sec:effect_of_cf_weighting}
In models of the periphery that include more low SR fibers at high frequencies, synaptopathic losses will change.
% section effect_of_cf_weighting (end)

\section{Effect of brainstem model} % (fold)
\label{sec:effect_of_brainstem_model}
Does our intuition about a more physiologicaly relevant brainstem and midbrain model capturing more of the diversity of human responses bear out? 
% section effect_of_brainstem_model (end)
\cleardoublepage{}

% -*- root: ../gvoysey-thesis.tex -*-
\chapter{Discussion}
\label{chapter:Discussion}
\thispagestyle{myheadings}

% set this to the location of the figures for this chapter. it may
% also want to be ../Figures/2_Body/ or something. make sure that
% it has a trailing directory separator (i.e., '/')!
\graphicspath{{6_Discussion/Figures/}}

This chapter compares the results obtained in this thesis with the human results obtained by \citeauthor{Mehraei2016Auditory}, and offers justifications and possible explanations for their similarities and differences.
\cleardoublepage{}

% -*- root: ../gvoysey-thesis.tex -*-
\chapter{Conclusion}
\label{chapter:Conclusion}
\thispagestyle{myheadings}

% set this to the location of the figures for this chapter. it may
% also want to be ../Figures/2_Body/ or something. make sure that
% it has a trailing directory separator (i.e., '/')!
\graphicspath{{7_Conclusion/Figures/}}

\section{Chapter Summary} % (fold)
\label{sec:chapter_summary}

% section chapter_summary (end)


\cleardoublepage{}

%==========================================================================%
% Bibliography
\newpage
\singlespace{}
\printbibliography[heading=bibintoc]{}
\cleardoublepage{}

%==========================================================================%
% Curriculum Vitae
\includepdf[pages={ - }]{0_Prelim/gvoysey-resume.pdf}

%==========================================================================%
\end{document}
 