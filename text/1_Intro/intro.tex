% -*- root: ../gvoysey-thesis.tex -*-
\chapter{Introduction}
\label{chapter:Introduction}
\thispagestyle{myheadings}
\section{Motivation}
The variability of overall performance between normal hearing listeners, particularly in super-threshold tasks performed in complex acoustic environments such as the cocktail party problem, has been recognized in the literature for many years.  Until recently, this variability was largely attributed to a broadly-defined ``Central Processing Disorder'' in the absence of Noise-Induced Hearing Loss (NIHL). The performance of the auditory periphery has been thought to be sufficiently characterized by audiometric threshold testing, as well as Distortion-Product Otoacoustic Emissions (DPOAEs) and ABR for finer-grained assessments of peripheral function.

\section{Implication of the auditory periphery in cochlear synaptopathy}
Recently, selective deafferentiation of low spontaneous rate fibers of the AN in the auditory periphery that do not affect audiometric thresholds have been convincingly demonstrated in mouse \citep{Kujawa2009Adding}, gerbil \citep{Furman2013NoiseInduced}, and recently, chinchilla (Liberman, unpublished); a growing body of psychophysical evidence suggests that a similar pathology occurs in humans \citep{Bharadwaj2015Individual}.  Synaptic damage at the hair cell in the Organ of Corti has been observed both in response to noise with intensities sufficient to induce a temporary threshold shift (TTS), which does not permanently affect thresholds or hair cell life, and due to age alone in quiet \citep{Sergeyenko2013AgeRelated,Fernandez2015Aging}. This phenomenon has been variously described as ``cochlear synaptopathy''\citep{Bharadwaj2014Cochlear}, ``auditory neuropathy'', or ``Hidden Hearing Loss''.  

It is now thought that selective low-SR loss is a hallmark of HHL.  Consequently, it has been implicated in performance degradation in cocktail party scenarios in normal-hearing listeners \citep{Bharadwaj2015Individual,Bharadwaj2014Cochlear}.  Unlike NIHL, no objective and noninvasive measure of HHL in humans has been established.  While work is ongoing in cadaveric studies, the relationship between low-SR damage and HHL in humans has relied on inference from a combination of ABR, DPOAE, and psychometric measures, and no direct measure has yet been demonstrated that specifically implicates low-SR fiber loss as a causative factor.

\section{Human psychophysical tests suggest a diagnostic measure}

Towards this goal of defining an objective measure of fiber loss, \cite{Mehraei2015Individual,Mehraei2016Auditory} have performed a series of experiments that relate psychophysical performance in a tone in noise detection task to measured latency changes in ABR wave V as a function of signal to noise ratio.  They hypothesized that the loss of low-SR/high-threshold AN fibers would contribute to a faster recovery time of the compound action potential of the AN.  In a perceptual task, this translates to higher thresholds, and faster threshold recovery. In a group of 28 NHT subjects, comparison of ABR data and psychoacoustic performance demonstrate a relationship consistent with an impairment in low-SR population response.

\section{Computational models of the periphery are not predictive}

While psychophysical experiments have supported the hypothesis of the importance of low-SR fibers, modeling the response of the auditory periphery, brainstem, and midbrain to the stimuli used in experiments has so far failed to produce results that align with experiment or intuition.

This thesis seeks to investigate the causes of this disparity, and proposes several novel additions to two auditory models to remedy it. 


