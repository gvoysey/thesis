% -*- root: ../gvoysey-thesis.tex -*-
\chapter{Literature Review}
\label{chapter:literaturereview}
\thispagestyle{myheadings}
% set this to the location of the figures for this chapter. it may
% also want to be ../Figures/2_Body/ or something. make sure that
% it has a trailing directory separator (i.e., '/')!
\graphicspath{{2_LiteratureReview/Figures/}}
\section{Chapter Summary} % (fold)
\label{sec:review_summary}
This chapter lays out a review of the relevant literature this thesis relies on. First, an overview of the clinical significance and relevant neuroanatomy of cochlear synaptopathy are given.  Then, a review of the computational models that will be used is presented. 

% section review_summary (end)
\section{Cochlear Synaptopathy} % (fold)
\label{sec:cochlear_synaptopathy}
% section cochlear_synaptopathy (end)
Deafferentiation is the loss of one or more synapses between an Inner Hair Cell (IHC) and its innervating spiral ganglia. 

\cite{Kujawa2009Adding} showed in noise exposed mice that significant deafferentiation can occur with no permanent changes in threshold tuning curves and no hair cell death. \autoref{fig:furman-2} shows that deafferentiation was confirmed histologically by triple-staining cross-sections of the Organ of Corti post noise exposure.  This reveals a synaptic loss. 

\begin{figure}[htbp]
	\centering
	\includegraphics[width=0.95\textwidth]{furman-2.pdf}
	\caption[Deafferentation of Inner Hair Cells]{Deafferentiation of IHCs precedes hair cell death in noise exposed mice.  Figure is reprinted from~\cite{Furman2013NoiseInduced}}
	\label{fig:furman-2}
\end{figure}


\cite{Sergeyenko2013AgeRelated} extended this work to demonstrate that this deafferentiation may also arise solely as a function of time.  In a study of mice aged 4 to 144 weeks that were never exposed to loud sounds, a similar loss of hair cell projections was observed.


\section{Physiology of the Auditory Nerve} % (fold)
\label{sec:physiology_of_the_auditory_nerve}
\subsection{Spontaneous Rates of Fibers} % (fold)
\label{sub:spontaneous_rates_of_fibers}
In the absence of stimulus, individual fibers of the auditory nerve exhibit a wide range of average firing rates: human AN fibers have spontaneous rates between 0--120 spikes/second.  Any individual fiber's spontaneous rate varies slowly over time, but will fall within a relatively narrow band.  The fibers of the auditory nerve are divided into two, or sometimes three, categories: low-, medium-, and high spontaneous rate (SR).  Different authors assign different maximum firing rates to each category:~\cite{Temchin2008Threshold} categorizes SRs below 18 spikes/second to be ``low/medium'', and anything above that to be ``high''.  Others, such as~\cite{Liberman1978AuditoryNerve}, define only two categories.  
% subsection spontaneous_rates_of_fibers (end)
\subsection{Low Spontaneous Rate Fibers Suffer Selective Losses} % (fold)
\label{sub:low_spontaneous_rate_fibers_suffer_selective_losses}
\cite{Furman2013NoiseInduced} and others have demonstrated that noise-induced cochlear synaptopathy is selective for low-SR fibers, particularly at high frequencies.  As shown in \autoref{fig:furman-5}, examination of fiber loss after acoustic trauma demonstrates a preferential loss of low-SR fibers, particularly above 4 kHz.  

\begin{figure}[htbp]
	\centering
	\includegraphics[width=0.95\textwidth]{furman-5.pdf}
	\caption[Synaptopathy is Selective]{Synaptopathy is selective for fibers with low spontaneous rates, and particularly selective for low-SR fibers at high frequencies. Figure reprinted from~\cite{Furman2013NoiseInduced}}
	\label{fig:furman-5}
\end{figure}

\section{Relevant Functional Neuroanatomy of the Auditory Midbrain} % (fold)
\label{sec:relevant_functional_neuroanatomy_of_the_auditory_midbrain}
\subsection{The Cochlear Nucleus} % (fold)
\label{sub:the_cochlear_nucleus}
The primary projection from the AN is the Cochlear Nucleus, an inhomogenous structure that is the first auditory relay station located in the ipsilateral medulla of the brainstem.  
\subsection{The Dorsal Cochlear Nucleus} % (fold)
\label{sub:the_dorsal_cochlear_nucleus}
\cite{Ryugo2008Projections} demonstrated in cat that low-SR fibers have a anatomical projection bias towards the small cap of the Dorsal Cochlear Nucleus.  While low-SR fibers project to many areas, the small cap receives input from low-SR fibers exclusively, suggesting a selective role for low-SR projections. \cite{Liberman1993Central} found similar results. 

Further, while projections are selective as shown in \autoref{fig:ryugo}, the projections have relatively shallow but broad arbors. This anatomical specificity of projection combined with a breadth of coverage supports a particular role for low-SR fibers, and does not rule out the possibility of further specificities in higher brainstem and midbrain areas. 

\begin{figure}[htbp]
	\centering
	\includegraphics[width=0.95\textwidth]{ryugo-1.pdf}
	\caption[Low SR Fibers Project to the Small Cap]{Low-SR fibers project to the small cap of the DCN.  From~\cite{Ryugo2008Projections}, this shows a lateral view of a low SR fiber as it collateralizes (red) in the rostral and lateral SCC (CF=0.45 kHz; SR=1.2 s/s; Th=34 dB SPL).}
	\label{fig:ryugo}
\end{figure}

While the VCN is critically important for the processing of binaural phenomena, the specificities of projection observed in the DCN suggest an interesting role for synaptopathic losses that may be more monaural.
% subsection the_ventral_cochlear_nucleus (end)
\subsection{The Inferior Colliculus} % (fold)
\label{sub:the_inferior_colliculus}
The IC has long been regarded as the last pre-thalamic obligate waystation for ascending auditory information, and a major center of pre-cortical auditory processing with many diverse functions \citep{Cant2005Atlas,Covey2008Inputs,Moore1985Projections}.~\cite{Beebe2016Extracellular} has recently found at least four morphologically different GABAergic neuron types in IC that react in different ways to auditory stimuli.
% subsection the_inferior_colliculus (end)
% section functional_neuroanatomy_of_the_auditory_midbrain (end)

\section{Models of the Auditory Periphery} % (fold)
\label{sec:models_of_the_auditory_periphery}
\subsection{The Verhulst Model} % (fold)
\label{sub:the_verhulst_model}
A functional model of the auditory periphery was developed by~\cite{Verhulst2015Functional}.  As outlined in \autoref{fig:verhulst-1}, the model consists of a middle ear preprocessing model adapted from~\cite{Meddis2010Computational}.  Input is passed to a cochlear transmission line model, which estimates BM displacements and velocities for an arbitrary number of BM sections (default: 1000).  Motions of the BM are translated into IHC bundle deflections and passed through a nonlinearity.  Estimates of the Instantaneous Firing Rate (IFR) are made by a method adapted from~\cite{Westerman1988Diffusion}, which implements a three-store diffusion model of synaptic vesicle and neurotransmitter release and reuptake.  

The version of the model given by~\cite{Verhulst2015Functional} also includes a CN and IC modeling stage from \cite{Nelson2004Phenomenological}, and the final model output are estimates of ABR Wave I, Wave III, and Wave V. 

\begin{figure}[htbp]
	\centering
	\includegraphics[width=0.95\textwidth]{verhulst-1.pdf}
	\caption[The Verhulst Model]{Major subunits of the Verhulst transmission line model of the auditory periphery.  Figure is reprinted from \cite{Verhulst2015Functional}.}
	\label{fig:verhulst-1}
\end{figure}


\subsection{The Zilany and Bruce Model} % (fold)
\label{sub:the_zilany_and_bruce_model}
\cite{Zilany2006Modeling} proposed a phenomenological, signals-driven model of the auditory periphery.  It has been refined and updated since to account for an increasing number of phenomena including estimations of speech intelligibility \citep{Zilany2007Predictions}, long-term IHC adaptation with power-law dynamics \citep{Zilany2009Phenomenological}, and updates to more closely model human parameters \citep{Zilany2014Updated}.  

The approach is outlined in \autoref{fig:zilany-1} and consists of a phenomenological power-law model that has filters for each stage of the periphery.  Fractional Gaussian noise is optionally added per-channel to simulate the stochasticities inherent in AN fiber spontaneous rates. 

\begin{figure}[htbp]
	\centering
	\includegraphics[width=0.95\textwidth]{zilany-1.pdf}
	\caption[The Zilany Model]{Major components of the Zilany model of the auditory periphery.  Figure reprinted from \cite{Zilany2009Phenomenological}}
	\label{fig:zilany-1}
\end{figure}
% subsection the_zilany_and_bruce_model (end)
\section{Models of the Auditory Midbrain and Brainstem} % (fold)
\label{sec:models_of_the_auditory_midbrain_and_brainstem}
\subsection{The Nelson-Carney Model} % (fold)
\label{sub:the_nelson_carney_model}
\cite{Nelson2004Phenomenological}

\subsection{The Carney Model} % (fold)
\label{sub:the_carney_model}

% subsection the_carney_model (end)
% subsection the_nelson_carney_model (end)
% section models_of_the_auditory_midbrain_and_brainstem (end)



\section{Candidate Objective Measures of Cochlear Synaptopathy} % (fold)
\label{sec:objective_measures_of_cochlear_synaptopathy}

% section objective_measures_of_cochlear_synaptopathy (end)
