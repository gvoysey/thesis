% -*- root: ../gvoysey-thesis.tex -*-
\chapter{Literature Review}
\label{chapter:literaturereview}
\thispagestyle{myheadings}
% set this to the location of the figures for this chapter. it may
% also want to be ../Figures/2_Body/ or something. make sure that
% it has a trailing directory separator (i.e., '/')!
\graphicspath{{2_LiteratureReview/Figures/}}
\section{Chapter Summary} % (fold)
\label{sec:review_summary}
This chapter lays out a review of the relevant literature this thesis relies on. First, an overview of the clinical significance and relevant neuroanatomy of cochlear synaptopathy are given.  Then, a review of the computational models that will be used is presented. 

% section review_summary (end)
\section{Cochlear Synaptopathy} % (fold)
\label{sec:cochlear_synaptopathy}
% section cochlear_synaptopathy (end)
Deafferentiation is the loss of one or more synapses between an Inner Hair Cell (IHC) and its innervating spiral ganglia. 

\cite{Kujawa2009Adding} showed in noise exposed mice that significant deafferentiation can occur with no permanent changes in threshold tuning curves and no hair cell death. Deafferentiation was confirmed histologically by double-staining.

\cite{Sergeyenko2013AgeRelated} extended this work to demonstrate that this deafferentiation may also arise solely as a function of time.  In a study of mice aged 4 to 144 weeks that were never exposed to loud sounds, a similar loss of hair cell projections was observed. 


\section{Physiology of the Auditory Nerve} % (fold)
\label{sec:physiology_of_the_auditory_nerve}
\subsection{Spontaneous Rates of Fibers} % (fold)
\label{sub:spontaneous_rates_of_fibers}
In the absence of stimulus, individual fibers of the auditory nerve exhibit a wide range of average firing rates.  
% subsection spontaneous_rates_of_fibers (end)
\subsection{Low Spontaneous Rate Fibers Suffer Selective Losses} % (fold)
\label{sub:low_spontaneous_rate_fibers_suffer_selective_losses}
\cite{Furman2013NoiseInduced} demonstrated 
% subsection low_spontaneous_rate_fibers_suffer_selective_losses (end)

\subsection{Current Controversies} % (fold)
\label{sub:current_controversies}
% subsection current_controversies (end)
% section physiology_of_the_auditory_nerve (end)

\section{Relevant Functional Neuroanatomy of the Auditory Midbrain} % (fold)
\label{sec:relevant_functional_neuroanatomy_of_the_auditory_midbrain}
\subsection{The Dorsal Cochlear Nucleus} % (fold)
\label{sub:the_dorsal_cochlear_nucleus}
\subsubsection{The Small Cap Area} % (fold)
\label{ssub:the_small_cap_area}

% subsubsection the_small_cap_area (end)
% subsection the_dorsal_cochlear_nucleus (end)
\subsection{The Ventral Cochlear Nucleus} % (fold)
\label{sub:the_ventral_cochlear_nucleus}

% subsection the_ventral_cochlear_nucleus (end)
\subsection{The Inferior Colliculus} % (fold)
\label{sub:the_inferior_colliculus}

% subsection the_inferior_colliculus (end)
% section functional_neuroanatomy_of_the_auditory_midbrain (end)

\section{Auditory Modeling Environments} % (fold)
\label{sec:auditory_modeling_environments}
\subsection{The Carney Models} % (fold)
\label{sub:the_carney_models}
\subsubsection{The Zilany and Bruce Models} % (fold)
\label{ssub:zilany_and_bruce_2009_2014}

% subsubsection zilany_and_bruce_2009_2014 (end)
\subsubsection{Modulation Transfer Functions in the Inferior Colliculus} % (fold)
\label{ssub:modulation_transfer_functions_in_the_inferior_colliculus}

% subsubsection modulation_transfer_functions_in_the_inferior_colliculus (end)
% subsection the_carney_models (end)

\subsection{The Verhulst Model} % (fold)
\label{sub:the_verhulst_model}

% subsection the_verhulst_model (end)

% section auditory_modeling_environments (end)

\section{Objective Measures of Cochlear Synaptopathy} % (fold)
\label{sec:objective_measures_of_cochlear_synaptopathy}

% section objective_measures_of_cochlear_synaptopathy (end)
