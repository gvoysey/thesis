% -*- root: ../gvoysey-thesis.tex -*-
% ABSTRACT
Cochlear Synaptopathy (CS) is an emerging topic of hearing research that focuses on peripheral pathologies which leave pure-tone audiometric thresholds (PTA) unchanged but significantly impair threshold-independent hearing performance. No noninvasive quantitative measure of CS yet exists in humans. Primary among the proposed mechanisms of CS is selective damage of low spontaneous rate (low SR) fibers of the auditory nerve (AN), yet quantification of this loss and the implications of this synaptopathy for the behavior of higher auditory areas remains poorly understood. Recent work has established a relationship between Auditory Brainstem Response (ABR) Wave V latencies, which is thought to reflect the relative contribution of low-SR fibers to the AN, and a psychophysical measure of CS in humans~\citep{Mehraei2016Auditory}.  However, current biophysical models do not fully account for the observed results.

To begin to address the discrepancies between these experiments and biophysical models of hearing, a new comprehensive modeling tool was developed which allows parametric exploration of modeling space and direct comparison between major models of the auditory nerve and brainstem. More sophisticated models of the midbrain and brainstem were incorporated into the new modeling tool. Incorporating recent anatomical and electrophysiological results, which suggest a varying contribution of low-SR fibers for different audible frequencies, further addresses modeling efficacy.
