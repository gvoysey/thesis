% -*- root: ../gvoysey-thesis.tex -*-
% ABSTRACT
Cochlear Synaptopathy (CS) is an emerging topic of hearing research that focuses on peripheral pathologies which leave audiometric thresholds unchanged but significantly impair threshold-independent hearing performance. Primary among the proposed mechanisms of CS is selective damage of low spontaneous rate (low SR) fibers of the auditory nerve (AN), yet no noninvasive quantitative measure of CS yet exists in humans.

This work aimed to explore the hypothesis that measuring changes in latency of Wave V of the Auditory Brainstem Response (ABR) can predict the magnitude of preferentially damaged contributions by low spontaneous rate fibers to the output of the auditory nerve. While steps towards establishing a relationship between Wave V latencies and a psychophysical measure of CS has been recently undertaken in human experiments~\citep{Mehraei2016Auditory}, current biophysical models do not fully account for the observed results.

To begin to address the discrepancies between these experiments and biophysical models of hearing, a new comprehensive modeling tool was developed which allows parametric exploration of modeling space and direct comparison between major models of the auditory nerve and brainstem. 

It is hypothesized that modeling deficits may be in part a consequence of the deficiencies of current brainstem and midbrain models, including those of the Inferior Colliculus (IC) and Lateral Lemniscus (LL), where Wave V is thought to originate.  To rectify those deficiencies, more sophisticated models of the midbrain and brainstem were incorporated into the new modeling tool. Incorporating recent anatomical and electrophysiological results, which suggest a varying contribution of low-SR fibers for different audible frequencies, further addresses modeling efficacy.
