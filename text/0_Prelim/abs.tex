% -*- root: ../gvoysey-thesis.tex -*-
% ABSTRACT
Hidden Hearing Loss (HHL) is an emerging topic of hearing research that focuses on peripheral pathologies which leave audiometric thresholds unchanged but significantly impair threshold-independent hearing performance. Primary among the proposed mechanisms of HHL is selective damage of low spontaneous rate (low SR) fibers of the auditory nerve (AN), yet no noninvasive quantitative measure of this mechanism yet exists in humans.

This work aimed to support the hypothesis that measuring changes in latency of Wave V of the Auditory Brainstem Response (ABR) can predict the magnitude of preferentially damaged contributions by low spontaneous rate fibers to the output of the auditory nerve. While the relationship between Wave V latencies and a psychophysical measure of HHL has been recently established, current biophysical models do not fully account for the observed results.

We hypothesize that this modeling deficit is largely a consequence of the deficiencies of current brainstem and midbrain models, particularly those of the Inferior Colliculus (IC) and Lateral Lemniscus (LL), where Wave V is thought to originate.  To rectify those deficiencies, more sophisticated models of the midbrain and brainstem were incorporated.  Nonlinear weighting of the auditory nerve response, as supported by recent anatomical work, was also incorporated.  To quantify the effects of these changes on modeling predictions, a comprehensive modeling tool was developed which allows parametric exploration of modeling space and direct comparison between major models of the auditory nerve and brainstem. 