% -*- root: ../gvoysey-thesis.tex -*-
\phantom{.}
\vspace{4in}

\begin{singlespace}
\begin{quote}
  We've heard a lot of models\ldots{}and heard suggested that we should take things out of our models to figure out what's important.  But in some sense, when I look at the diversity of models that have been presented so far---each of us leave \emph{out} things.  So maybe in some sense we've got a start towards that approach.\\  
  
  So I can ask this question two ways, but let me ask it this way: \emph{What should we leave in?} What's the bare minimum we should leave in as we try to understand what's important about the function of the cochlea?
    % \textit{Facilis descensus Averni;}\\
  % \textit{Noctes atque dies patet atri janua Ditis;}\\*
  % \textit{Sed revocare gradum, superasque evadere ad auras,}\\
  % \textit{Hoc opus, hic labor est.}\hfill{Virgil (from Don's thesis!)}
  \vspace{2.5em}
  \begin{flushright}
  \textit{David C. Mountain\\Mechanics of Hearing (Attica, Greece 2014)}
  \end{flushright}
\end{quote}
\end{singlespace}
  

% \vspace{0.7in}
%
% \noindent
% [The descent to Avernus is easy; the gate of Pluto stands open night
% and day; but to retrace one's steps and return to the upper air, that
% is the toil, that the difficulty.]
