% -*- root: ../gvoysey-thesis.tex -*-
\chapter{Results}
\label{chapter:Results}
\thispagestyle{myheadings}

% set this to the location of the figures for this chapter. it may
% also want to be ../Figures/2_Body/ or something. make sure that
% it has a trailing directory separator (i.e., '/')!
\graphicspath{{5_Results/Figures/}}
\section{Chapter Summary} % (fold)
\label{sec:results_summary}
This chapter describes the results obtained when using the modeling environment described in~\autoref{chapter:Methods}.
% section results_summary (end)
\section{Tone in noise} % (fold)
\label{sec:tone_in_noise}
The model ABR in response to a click-train in noise was computed for a variety of masker ratios.   
The experiment design tool described in \autoref{sec:automated_parameter_exploration} was used to specify a range of values for each parameter in Corti to reveal the relative contributions of each. 

\subsection{Stimuli} % (fold)
\label{sub:stimuli}
Following \citeauthor{Mehraei2015Auditory,Mehraei2016Auditory}, 6 stimului were programmatically generated and stored as WAV files with a sampling frequency of 100 kHz.  As shown in \autoref{fig:stimuli-used}, stimulus onset was delayed by 20 $\mu$s of silence, and then consisted of 80 dB SPL clicks with a repetition rate of 100 ms in the presence of gaussian noise at different signal to noise ratios. 

Importantly, these stimuli are all well above the threshold of audibility; the aim is to obtain a response of the auditory models as they react robustly to a clearly audible input. 

\begin{figure}[htbp]
	\centering
	\includegraphics[width=0.95\textwidth]{stimuli-used.pdf}
	\caption[Experimental Stimuli]{Stimuli used to drive the auditory models.}
	\label{fig:stimuli-used}
\end{figure}

All other parameters--choice of peripheral and brainstem model and logistically weighted fiber distributions---were fully explored.

In total, 240 separate simulations were run in parallel on Boston University's high-performance computing cluster over the course of approximately 9 days.  Model output was automatically stored into a HDF5 database approximately 250 GB in size.

\section{Effect of Synaptopathy} % (fold)
\label{sec:effect_of_synaptopathy}
The effects of four types of synaptopathy---moderate, severe, low-SR specific moderate, and low-SR specific severe---were simulated, with the synaptic degradation parameters as given in \autoref{fig:synaptopathy}. 

\begin{figure}[htbp]
	\centering
	\includegraphics[width=\textwidth]{synaptopathy-all.pdf}
	\caption[Effects of Synaptopathy]{The effects of varying levels of synaptopathy on model responses to a stimulus at multiple noise levels.  Some traces (top) are present but not visible as they precisely overlay other results.}
	\label{fig:synaptopathy_results}
\end{figure}

The effect of fiber loss on Wave V peak latency and Wave I peak amplitude are given in \autoref{fig:synaptopathy_results}.  Consistent with prediction, Wave I amplitudes are decreased as a function of synaptopathy, as well as a function of increasing noise masker level.  Wave V latencies exhibit a decrease in latency but non-monotonically, and only for some masker levels.
% section effect_of_synaptopathy (end)

\section{Effect of peripheral model} % (fold)
\label{sec:effect_of_peripheral_model}
Because the response trends vary quite little as a function of the types of synaptopathy modeled in \autoref{sec:effect_of_synaptopathy}, the relative effects of model types can be explored while holding the type of synaptopathy fixed. 

The effects of peripheral model choice on Wave V peak latency and Wave I peak amplitudes are given in \autoref{fig:periphery_results}.   In general, the Verhulst model predicts both larger Wave V latencies and larger changes as a function of SNR compared to the Zilany model, which predicts generally small changes.  Wave I amplitude estimations follow similar shapes.  While both produce physiologically plausible responses, the Verhulst model predicts amplitudes approximately half the magnitude of the Zilany model. 

\begin{figure}[htbp]
	\centering
	\includegraphics[width=\textwidth]{periphery-all.pdf}
	\caption[Effects of Peripheral Models]{Effects of Peripheral Models}
	\label{fig:periphery_results}
\end{figure}

% section effect_of_peripheral_model (end)


\section{Effect of CF weighting} % (fold)
\label{sec:effect_of_cf_weighting}
\begin{figure}[htbp]
	\centering
	\includegraphics[width=\textwidth]{weighting-all.pdf}
	\caption[Effects of CF Weighting]{Effects of CF Weighting}
	\label{fig:cf_results}
\end{figure}

The effects of logisitically weighing the fiber type distribution along the basilar membrane is given in \autoref{fig:cf_results}.  Surprisingly, there was very little relative effect with the Verhulst model.  The Zilany model showed consistent but relatively small elevation of Wave V latency and suppression of Wave I amplitudes. 
% section effect_of_cf_weighting (end)

\section{Effect of brainstem model} % (fold)
\label{sec:effect_of_brainstem_model}
\begin{figure}[htbp]
	\centering
	\includegraphics[width=\textwidth]{brainstem-all.pdf}
	\caption[Effects of Brainstem Models]{Effects of Brainstem Models.  Wave 1 amplitudes are unaffected by the brainstem model.}
	\label{fig:brainstem_results}
\end{figure}

The effects of a more complex brainstem model is given in \autoref{fig:brainstem_results}.  No effects are observed for Wave I amplitudes, as Wave I originates at the level of the auditory nerve.  Only slight elevations of Wave V latencies are observed.  
% section effect_of_brainstem_model (end)